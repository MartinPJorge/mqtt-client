\documentclass{upmassignment}
\usepackage[spanish]{babel}
\usepackage{ifthen}
\usepackage{amsmath}
\usepackage{amsfonts}
\usepackage{booktabs}
\usepackage[x11names]{xcolor}
\usepackage{tcolorbox}
\usepackage{cclicenses}

\usepackage{listings}
\lstset{basicstyle=\ttfamily,
  showstringspaces=false,
  commentstyle=\color{red},
  keywordstyle=\color{blue},
  backgroundcolor=\color{gray!30},
}


\usetikzlibrary{calc}



% Para mostrar/ocultar soluciones
\newboolean{show}
\setboolean{show}{true}
\setboolean{show}{false}
\usepackage{environ}
\NewEnviron{solucion}{
  \ifshow
      \begin{answer}\BODY\end{answer}
  \fi}






\coursetitle{Redes y Servicios}
\courselabel{RSER}
\exercisesheet{Cliente MQTT}{}
\student{\ }%
\semester{Primer Semestre 2025/2026}
\date{\today}
\university{Universidad Politécnica de Madrid}
\school{Departamento de Ingeniería de Sistemas Telemáticos}
%\usepackage[pdftex]{graphicx}
%\usepackage{subfigure}


\setlength{\textwidth}{5.0in}
\linespread{1.3}
\renewcommand{\PB}{{\bfseries Problema}}



\begin{document}

\section*{Introducción}

\noindent
En esta práctica\footnote{Todas las preguntas
tienen el mismo valor/puntuación.
Cada apartado del Problema~2 tiene el
mismo valor que el resto de Problemas.}\footnote{Este material está protegido por la
licencia
CC BY-NC-SA 4.0.
\byncsa
}
vamos a crear un cliente
MQTT
\texttt{client.py}
que enviará constantes vitales
de un Sensor.
Para ello se proporciona el programa
\texttt{reporter.sh}
(disponible en Moodle)
que va añadiendo líneas a un archivo
\texttt{report.csv}
que almacena las constantes vitales
de un sensor.






\begin{figure}[h]
    \centering
\begin{tikzpicture}
    \node[draw,rectangle] (sensor) at (0,0)
        {Sensor};
    \node[draw,rectangle] (script)
        at ($(sensor)+(0,-1)$)
        {\texttt{reporter.sh}};
    \node[draw,rectangle,anchor=west]
        (report)
        at ($(script.east)+(1,0)$)
        {\texttt{report.csv}};
    \node[draw,rectangle]
        (utils)
        at ($(report)+(0,1)$)
        {\texttt{utils.py}};
    \node[draw,rectangle,fill=gray!30]
        (client)
        at ($(utils)+(0,1)$)
        {\texttt{client.py}};
    \node[draw,rectangle]
        (server)
        at ($(client)+(5,0)$)
        {server};
    \node[align=center,anchor=north]
        at (server.south)
        {\texttt{broker.emqx.io}\\
        \texttt{port:1883}};

    \draw[arrows=<->] (sensor.south)
        -- (script.north);
    \draw[arrows=->] (script.east)
        -- (report.west);
    \draw[arrows=<->] (report.north)
        -- (utils.south);
    \draw[arrows=<->] (utils.north)
        -- (client.south);
    \draw[arrows=<->] (client.east)
        --
        node[midway,above,align=center]
        {MQTT\\connection}
        (server.west);



\end{tikzpicture}
\caption{Escenario de la práctica.}
\label{fig:escenario}
\end{figure}


\section*{Instalación de Dependencias}
\noindent
Antes de comenzar la práctica debemos
instalar las dependencias necesarias
para python y MQTT. 
Descargue el fichero
\texttt{practica-cliente.zip}
de Moodle y descomprímalo en
su carpeta personal.
Abra una terminal y ejecute
las siguientes líneas
\begin{lstlisting}[language=bash]
$ cd ~/practica-cliente
$ ./dependencies.sh
\end{lstlisting}





\section*{Generación de Constantes Vitales}
\noindent Para generar el archivo
\texttt{report.csv} con las constantes
vitales, ejecute el programa
que reporta las medidas del sensor:
\begin{lstlisting}[language=bash]
$ cd ~/practica-cliente
$ ./reporter.sh
\end{lstlisting}
Tras ejecutar las líneas de arriba,
verá que se ha generado un archivo
\texttt{report.csv} con las siguientes
columnas (separadas por comas):
\begin{lstlisting}
Idx, Time (s), HR (BPM), RESP (BPM), SpO2 (%),TEMP (*C),OUTPUT
0,0,94,21,97,36.2,Normal
1,1,94,25,97,36.2,Normal
2,2,101,25,93,38,Abnormal
3,3,55,11,100,35,Abnormal
\end{lstlisting}
\emph{Nota}:
el archivo \texttt{report.csv} se genera
de cero para cada ejecución del
el \texttt{reporter.sh}. 


\section*{Lectura de Constantes Vitales}
\noindent 
A continuación va a programar la
función \texttt{read\_report()} encargada
de leer constantes vitales del reporte.
Esta función se encuentra en el archivo
\texttt{utils.py} y se encarga
de leer el archivo de reporte
\texttt{report.csv}. Por ejemplo,
podemos empezar en la décima línea
del archivo y leer dos líneas usando
\begin{lstlisting}[language=bash]
$ python3
>>> import utils
>>> utils.read_report('report.csv',10,2)
['9,9,94,26,97,29,Normal\n', '10,10,94,26,97,42,Abnormal\n']
\end{lstlisting}

Para más detalles sobre el funcionamiento
de \texttt{read\_report()}, lea
la descripción de la función en el
archivo \texttt{utils.py}.
Tendrá que utilizar las funciones
\texttt{open()} y \texttt{read\_lines()}
para programar la función.


\begin{problemlist}
    \pbitem Programe la función
    \texttt{read\_report()} y
    ejecútela usando como parámetros
    \texttt{LAST\_LINE=$X$}
    y
    \texttt{num\_lines=$\lceil \tfrac{X}{2} \rceil$},
    donde $X$ es el número de grupo
    asignado por el profesor.
        Nótese que la muestra $X$
        se refiere al campo \texttt{Idx}
        $X$.
    Copie el resultado en el
    campo \texttt{lectura} del
    archivo
    \texttt{respuestas-X.json}
    usando dobles comillas:
\begin{lstlisting}
{
    "lectura": [ "9,9,94,26,97,29,Normal\n",
      "10,10,94,26,97,42,Abnormal\n"]
}
\end{lstlisting}
\end{problemlist}
\emph{Nota}: deje corriendo
\texttt{reporter.sh} para que genere
suficientes reportes.



\section*{Cliente MQTT}
\subsection*{Conexión}
\noindent
A continuación va a programar parte
de la lógica del cliente MQTT.
En concreto se va a programar la conexión
del \texttt{client.py} con el
server ilustrada en la
Figura.~\ref{fig:escenario},
donde se especifica la dirección y puerto
del servidor.

Para la comunicación con el servidor
mediante MQTT,
el programa \texttt{client.py} utiliza
la librería \texttt{PAHO} de python.
Los pasos para abrir una conexión
(consulte la plantilla del cliente
\texttt{client.py})
son los siguientes:
\begin{enumerate}
    \item Conectarse con el servidor
        MQTT: función
        \texttt{connect\_mqtt()}; e
    \item invocar el bucle de publicación
        de mensajes:
        función \texttt{publish()}.
\end{enumerate}

\begin{figure}[h]
    \centering
    \begin{tikzpicture}
        \node[draw,rectangle,fill=gray!30]
            (client) at (0,0)
            {\texttt{client.py}};
        \node[draw,rectangle] (server) at
            ($(client)+(8,0)$)
            {server};
        \node[align=center,anchor=north]
            at (server.south)
            {\texttt{broker.emqx.io}\\
            \texttt{port:1883}};

        \draw[arrows=<->]
            (client.north east)
            --
            node[midway,above]
            {\texttt{connect\_mqtt()}}
            (server.north west);
        \draw[arrows=->]
            (client.south east)
            --
            node[midway,below]
            {\texttt{publish()}}
            (server.south west);
    \end{tikzpicture}
    \caption{Conexión y publicación del
    cliente.}
    \label{fig:connect-publish}
\end{figure}


Modifique la plantilla de
\texttt{client.py} para conectarse
al servidor y responda a las siguientes
preguntas.

\begin{problemlist}
    \stepcounter{enumi}
    \pbitem Inicie una captura wireshark
        en todas las interfaces
        (\texttt{any}) poniendo filtro
        ``mqtt''. Ejecute en la consola
        \texttt{client.py} e identifique
        las tramas de conexión y ACK
        en wireshark. Detenga la captura
        y rellene las
        siguientes preguntas en el
        JSON de respuestas:
        \begin{enumerate}
            \item ¿Cuál es el ID de su cliente?
            \item En la captura wireshark,
                ¿cuál es el tipo de mensaje
                MQTT
                del ACK de la conexión?
        \end{enumerate}
        Responda rellenando lso campos
        \texttt{clientid} y
        \texttt{ackmessagetype} del JSON
        de respuestas, respectivamente.\\
        Guarde\footnote{Para guardas solo los paquetes MQTT
                basta con poner en el filtro
                ``mqtt'',
                pinchar en
                File-Export Specified Packets,
                y en el cuadro ``Packet
                Range'' pinchar en la
                bola  ``Displayed''.}
                la captura en el fichero
        \texttt{captura-connect-grupoX.pcapng}.
Siga el mismo procedimiento del
pie de página para todas las capturas
de la práctica.
\end{problemlist}



\subsection*{Publicación}
\noindent A continuación debe
modificar el \texttt{client.py} para
leer los datos del sensor
(archivo \texttt{report.csv}) y publicarlos
en el topic
\texttt{rserGX/vitals}, donde X
es su número de grupo.

La función \texttt{publish()}
contiene un bucle que continuamente publica
mensajes MQTT. Modifique el bucle
para que espere
$(X\mod{5}) + 5$ segundos al final de cada
iteración.
Además, modifique \texttt{publish()}
para invocar a \texttt{read\_report()} y
publicar \emph{una a una}
las constantes vitales.
%% En concreto,
%% debe enviar el valor de
%% la columna \texttt{Time (s)}
%% (columna número 1)
%% y la columna
%% $(X \mod{4}) + 2$.



Tras haber realizado las modificaciones
indicadas, inicie una captura en
wireshark filtrando paquetes MQTT y
responda a las siguientes
preguntas usando la captura entregada.
Para ello responda en
en el JSON de respuestas a:


\begin{problemlist}
    \stepcounter{enumi}
    \stepcounter{enumi}
    \pbitem ¿En qué instante
        se envía la muestra $X$?
        (valor ``Time'' en wireshark).
        Nótese que la muestra $X$
        se refiere al campo \texttt{Idx}
        $X$.

    \pbitem ¿Cuánto tiempo
        pasa entre dos 
        Ping de MQTT?
\end{problemlist}
Responda rellenando lso campos
\texttt{instantemuestraX} (no sustituya
la X por su número de grupo) y
\texttt{tiempopings} del JSON
de respuestas, respectivamente.

Guarde la captura en el fichero
\texttt{captura-publish-grupoX.pcapng}.


\subsection*{Desconexión y QoS}
\noindent A continuación vamos a probar
los distintos niveles de QoS ofrecidos
por MQTT. Para ello basta con
añadir el argumento
\texttt{qos=$q$} a la función
\texttt{client.publish()}, donde
$q\in\{0,1,2\}$ especifica el nivel
QoS de MQTT.

En primer lugar vamos a probar
qué sucede con un Qos de 0 si se
cae la conexión. Para ello vamos a
apagar la interfaz
\texttt{eth0} usando el
siguiente comando
\begin{lstlisting}[language=bash]
$ ip link set down dev eth0
\end{lstlisting}
y la encenderemos con el siguiente comando
\begin{lstlisting}[language=bash]
$ ip link set up dev eth0
\end{lstlisting}



\subsection*{QoS 0}
\noindent
Ponga un QoS 0 y ejecute el
\texttt{client.py} usando el
siguiente comando para guardar la salida:
\begin{lstlisting}[language=bash]
$ python3 client.py 2>&1 | tee /tmp/qos0-grupoX.log
\end{lstlisting}

Inicie una
captura en wireshark filtrando
tráfico MQTT y espere a que
se envíen, al menos, dos tramas
PUBLISH. Apague la interfaz
\texttt{eth0} y espere a que
el cliente imprima
``Failed to send message [...]''.
Vuelva a encender la interfaz.


Responda a las siguientes preguntas
en el JSON de respuestas:
\begin{problemlist}
    \stepcounter{enumi}
    \stepcounter{enumi}
    \stepcounter{enumi}
    \stepcounter{enumi}
    \pbitem Especifique el \verb|Idx|
        de las muestras identificadas
        como perdidas por el cliente.
    \pbitem Especifique el \verb|Idx|
        de las muestras que no se
        han enviado con éxito.
\end{problemlist}
Responda rellenando los campos
\texttt{perdidascliente} y
\texttt{perdidas} del JSON
de respuestas, respectivamente.

Detenga la captura de wireshark y
guárdela en el fichero
\texttt{qos0-grupoX.pcapng}.


\subsection*{QoS 1}
\noindent
Ponga un QoS 1 y ejecute el \texttt{client.py}
como en el apartado anterior, esta vez
guardando el log en el archivo
\texttt{qos1-grupoX.log}.

De nuevo, inicie una captura en wireshark
filtrando tráfico MQTT y espere a que
se envíen, al menos, dos tramas PUBLISH.
Apague la interfaz \texttt{eth0}
y espere de nuevo a que el cliente
imprima ``Failed to send message [...]''.
Vuelva a encender la interfaz.





\begin{problemlist}
    \setcounter{enumi}{6}
    \pbitem Indique el
        \texttt{Message Identifier}
        de los mensajes duplicados.


    \pbitem Repita otra captura
        incrementando el \texttt{keepalive}
        al doble e indique el
        \texttt{Message Identifier}
        de los mensajes perdidos.

\end{problemlist}
Responda rellenando los campos
\texttt{duplicates} y
\texttt{duplicatesx2} del JSON
de respuestas, respectivamente.

Guarde ambas capturas
en los ficheros
\texttt{qos1-grupoX.pcapng} y\\
\texttt{qos1-x2keepalive-grupoX.pcapng}.




\section*{Entrega}
\noindent Se subirá a moodle un archivo
\texttt{clienteX.zip}
(con \texttt{X} el número de grupo)
que contenga:
\begin{enumerate}
    \item el cliente \texttt{client.py};
    \item el archivo \texttt{utils.py};
    \item el JSON de respuestas \texttt{respuestas-X.json};
    \item las trazas de Wireskark
        \texttt{captura-connect-grupoX.pcapng},\\
        \texttt{captura-publish-grupoX.pcapng},
        \texttt{qos0-grupoX.pcapng},\\
        \texttt{qos1-grupoX.pcapng},\
        \texttt{qos1-x2keepalive-grupoX.pcapng}; y
    \item los logs
        \texttt{qos0-grupoX.log},
        \texttt{qos1-grupoX.log}.
\end{enumerate}

\begin{tcolorbox}
    \textbf{Atención I}: las capturas
    deben contener
    \emph{solamente} tráfico MQTT.\\
    \textbf{Atención II}: una entrega
    sin los archivos especificados,
    o con archivos sin formato especificado
    tendrá un 0 en los \textbf{Problemas}
    correspondientes.
\end{tcolorbox}


\end{document}
